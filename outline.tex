\documentclass[paper=a4, fontsize=11pt]{scrartcl} % A4 paper and 11pt font size

\usepackage[english]{babel} % English language/hyphenation
\usepackage{amsmath,amsfonts,amsthm} % Math packages
\usepackage{sectsty} % Allows customizing section commands
\allsectionsfont{\centering \normalfont\scshape} % Make all sections centered, the default font and small caps

\usepackage{fancyhdr} % Custom headers and footers
\usepackage{bm}
\usepackage{upgreek}
\usepackage{tikz}
\usetikzlibrary{shapes, arrows}

\pagestyle{fancyplain} % Makes all pages in the document conform to the custom headers and footers
\fancyhead{} % No page header - if you want one, create it in the same way as the footers below
\fancyfoot[L]{} % Empty left footer
\fancyfoot[C]{} % Empty center footer
\fancyfoot[R]{\thepage} % Page numbering for right footer
\renewcommand{\headrulewidth}{0pt} % Remove header underlines
\renewcommand{\footrulewidth}{0pt} % Remove footer underlines
\setlength{\headheight}{13.6pt} % Customize the height of the header

%-------------------------------
%	TITLE SECTION
%-------------------------------

\newcommand{\horrule}[1]{\rule{\linewidth}{#1}} % Create horizontal rule command with 1 argument of height

\title{
\normalfont \normalsize
\textsc{Brigham Young University} \\ [25pt] % Your university, school and/or department name(s)
\horrule{0.5pt} \\[0.4cm] % Thin top horizontal rule
\huge Cool Title \\ % The assignment title
\horrule{2pt} \\[0.5cm] % Thick bottom horizontal rule
}

\author{authors} % Your name

\date{\normalsize\today} % Today's date or a custom date

\begin{document}

\maketitle % Print the title


\abstract{Can we fly 45 mph through a simulated environment using deep learning?}

%-------------------------------------------------
%	PROBLEM 1
%-------------------------------------------------


\section{Introduction}
\begin{itemize}
	\item Obstacle Avoidance
	\begin{itemize}
		\item Most obstacle avoidance algorigthms have focused on designing a closed-form avoidance rule given specific inputs.
		\item FFLOA - first ROAP, uses a heavy, specialized sensor, awesome results.
		\item Schopferer - closest to optimal ROAP.  Considers kinematic feasibility of avoidane - What sensor did it use?
		\item Oleynikova - Stereo vision FFLOA
		\item Saunders - Local memory Voxel Grid-based planning
		\item Some newer avoidance methods have leveraged the work in deep learning to come up with other obstacle avoidance methods.
		\item Ross - Supervised Learning obstacle avoidance from manually extracted features - read citations and cited by's
	\end{itemize}
\end{itemize}

\begin{itemize}
	\item Neural Networks
	\begin{itemize}
		\item Convoultion NN
		\item Deep NN
		\item Relevant Applications
	\end{itemize}
\end{itemize}

\section{Approach}

\begin{itemize}
	\item Simulation Description
	\begin{itemize}
		\item Using Gazebo
		\item Using 2nd-order dynamic model of roll and pitch with specified time constants $\zeta=0.707$, $\omega=???$ \textbf{figure out the appropriate response frequency}
		\item Randomly Generated Environments - define environment
	\end{itemize}
	\item Supervised Learning Beginning
	\begin{itemize}
		\item Use RRT to plan optimal trajectory
		\item Use a 2nd-order simplified model of Multirotor dynamics in simulation
		\item Learned to approximate the RRT plan through environment
		\item Train NN to extract optimized features for obstacle avoidance
		\item warm start perception and avoidance network for reinforcement learnin
	\end{itemize}
	\item Reinforcement Learning
	\begin{itemize}
		\item Cost function is deviation from path weighted with staying away form obstacles
		\item Eventually learned to 
	\end{itemize}
\end{itemize}






\end{document}
