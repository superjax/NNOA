\documentclass[paper=a4, fontsize=11pt]{scrartcl} % A4 paper and 11pt font size

\usepackage[english]{babel} % English language/hyphenation
\usepackage{amsmath,amsfonts,amsthm} % Math packages
\usepackage{sectsty} % Allows customizing section commands
\allsectionsfont{\centering \normalfont\scshape} % Make all sections centered, the default font and small caps

\usepackage{fancyhdr} % Custom headers and footers
\usepackage{bm}
\usepackage{upgreek}
\usepackage{tikz}
\usetikzlibrary{shapes, arrows}

\pagestyle{fancyplain} % Makes all pages in the document conform to the custom headers and footers
\fancyhead{} % No page header - if you want one, create it in the same way as the footers below
\fancyfoot[L]{} % Empty left footer
\fancyfoot[C]{} % Empty center footer
\fancyfoot[R]{\thepage} % Page numbering for right footer
\renewcommand{\headrulewidth}{0pt} % Remove header underlines
\renewcommand{\footrulewidth}{0pt} % Remove footer underlines
\setlength{\headheight}{13.6pt} % Customize the height of the header

%-------------------------------
%	TITLE SECTION
%-------------------------------

\newcommand{\horrule}[1]{\rule{\linewidth}{#1}} % Create horizontal rule command with 1 argument of height

\title{
\normalfont \normalsize
\textsc{Brigham Young University} \\ [25pt] % Your university, school and/or department name(s)
\horrule{0.5pt} \\[0.4cm] % Thin top horizontal rule
\huge Cool Title \\ % The assignment title
\horrule{2pt} \\[0.5cm] % Thick bottom horizontal rule
}

\author{authors} % Your name

\date{\normalsize\today} % Today's date or a custom date

\begin{document}

\maketitle % Print the title


\abstract{abstract goes here}

%-------------------------------------------------
%	PROBLEM 1
%-------------------------------------------------


\section{Introduction}

\begin{itemize}
	\item high-speed obstacle avoidance is a difficult problem
	\item Several attempts to solve the problem have been attempted
	\item it remains an open problem
	\item Deep learning is a developing field with wide applications
	\item Could potentially be used to solve obstacle avoidance
	\item Would require enourmous amounts of training data
	\item We propose to use simulated training data to augment real data
	\item Would require much less time, but network has to be general enough to use in real life
\end{itemize}

\section{Background}

\begin{itemize}
	\item OA Background
	\begin{itemize}
		\item There have been a number of attempts to solve the obstacle avoidance problem
		\item FFLAO, CEPA, CMU learning, etc...
		\item These are all based on algorithms, some have been tuned with training data
		\item We use a neural network to find the function, rather than manually developing one.
	\end{itemize}
\end{itemize}

\begin{itemize}
	\item  NN Background
	\begin{itemize}
		\item Neural Networks can be used to process camera data
		\item They have demonstrated a lot of effectiveness at playing video games
		\item Brief overview of supervised learning and it's applications
		\item They can be used to train an obstacle avoidance algorithm
	\end{itemize}
\end{itemize}

\begin{itemize}
	\item SL background
	\begin{itemize}
		\item Simulated training data is inherently different than real data
		\item People who have attempted to merge the two
		\item We have successfully augmented our real life training data with simulated data.
	\end{itemize}
\end{itemize}

\section{Implementation}
\begin{itemize}
	\item 
\end{itemize}







\end{document}
